\begin{table*}
\small
\renewcommand{\arraystretch}{0.7}
\caption[]{Summary of Radial Velocity Stellar Activity Sources}
\label{table:activity}
\begin{tabular}{cccl}
  \hline \\ [-1ex]
  \textbf{Activity} & \textbf{Typical} & \textbf{Typical Signal} & \textbf{Notes} \\
  \textbf{Source} & \textbf{Timescale} & \textbf{Amplitude} & \\
  \hline
  Flares & a few minutes & tens of \mps{} & Has distinct photometric and spectroscopic signatures. \\ 
  &&&Observations during a flaring event should be excluded \\
  &&& from planet searches$^1$. \\
  \hline
  Oscillations & tens of minutes & few \mps{} & Can be mitigated with sufficiently long exposure times \\
  &&&or multiple observations per night$^2$. \\
  \hline
  Granulation & tens of minutes & few \mps{} & See oscillations.  \\
  \hline
  Active Regions & few stellar & sub-\mps{} $\to$ & Timescale and amplitude depend heavily on the active \\
  & rotation periods & tens of \mps{} & region sizes and on stellar rotaton$^3$. See Fig.~\ref{fig:rotation} for \\
  &&& the distribution of M dwarf \prot{.} \\
  \hline
  Magnetic Cycles & $\sim  7-30$ years & tens of \mps{} & Is only important when searching for \\
  &&&long period planets.
  \end{tabular}
\begin{list}{}{}
\item {\bf{Notes.}}
      $^{1}$ \cite{reiners09}, $^2$ \cite{dumusque11a}, $^3$ \cite{dumusque11b}. \\
\end{list}
\end{table*}
