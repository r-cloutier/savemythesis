\chapter{Conclusions and Future Work}

This thesis has focused on the development and implementation of
semi-parametric Gaussian process (GP) regression modelling of stellar activity
and other nuissance signals. The GP formalism helps enable the detection as
well as the precise and accurate characterization of exoplanetary systems
around low mass stars in particular. \\

In Chapter~\ref{chap:BS} I simulated the
now ongoing SPIRou Legacy Survey-Planet Search (SLS-PS)
in which I search for planets
in synthetic RV time series using a GP treatment of activity to detect
those planets and calculate the expected planet yield of the survey. In
Chapter~\ref{chap:rvfc} I derive an analytical framework to calculate the
RV observational requirement to measure the masses of transiting planets at
a desired level of precision in the presence of correlated RV residuals stemming
from stellar activity and treated as a quasi-periodic GP as has largely become
conventional within the field. In Chapter\ref{chap:k218} the GP formalism is
applied to empirical time series from K2, HARPS, and CARMENES of the transiting
planetary system around the nearby mid-M dwarf K2-18. Lastly,
Chapter\ref{chap:orion} extends the one-dimensional GP regression modelling to
the search for transiting planets in high cadence light curves from TESS. Here
GP modelling was used to treat stellar varaibaility and residual systematic
effects while searching for repeating transit-like events in the 2 minute
extracted light curves. \\

Each of the applications of the GP formalism in the aforementioned chapters
may lead to a number of improvements or continued paths of investigation which
are discussed below.

\section{On the Accuracy of the SLS-PS Planet Yield Predictions}
Although not a critical assessment, in four years time it would be interesting
to compare the actual yield of the SLS-PS to the predictions presented in
Chapter~\ref{chap:BS}. The details of making this comparison one-to-one will
need to be worked out as the actual time series obtained during the SLS-PS will
certainly differ from those simulated both in terms of size, sampling, and
measurement uncertainties. This may or may not prove to be a major hinderance
however because it was determined through comparsion of many flavours of the
simulated SLS-PS that for \nrv{} $\gtrsim 50$, that the GP treatment of stellar
activity results in nearly white RV residuals such that the detection
sensitivity within the survey scales approximately as $\sqrt{N_{\text{RV}}}$ and
as $\sigma_{\text{RV}}^{-1}$. \\

Discrepancies between the predicted and actual planet yields could stem
from any of a number of inconsistencies. The assumed planet occurrence rates
throughout were derived from Kepler statistics and converted to planetary masses
using an empirical mass-radius relation. Inaccuracies from either calculation
in general or their inapplicability to the stars in the SLS-PS target list may
lead to a misestimated planet yield prediction. For example, the input Kepler
statistics were derived from late K to early M dwarfs \citep{dressing15a} and
may not be applicable to the full range of M dwarf spectral types targeted in
the SLS-PS. Another danger with simulated time series is that injected signals
may be too simplistic or that the number of injected signals is incomplete. For
example, the inclusion of systematic RV noise sources may directly affect the
survey's detection sensitivity. Another major source of potential inconsistency
is that the nature of detecting planets in the SLS-PS was based on an automated
detection algorithm intended to emulate the steps taken by a human when
searching for RV planets. Indeed the data flow from RV surveys is sufficently
small that manual planet detection is managable (unlike in large transit
surveys) and variance between human performance and the automated detection
algorithm from \cite{cloutier18a} may result in a different planet yield than
was initially predicted. Overall, it would be an interesting exercise to
isolate which of these sources contributes to the observed differences between
the predicted planet yield and the true planet yield following the conclusion
of the SLS-PS.

\section{Foreseen Improvements to the \texttt{RVFC}}
Recall that the Radial Velocity Follow-up Calculator (\texttt{RVFC}) is a tool
used to calculate the number of RV measurements required to measure a transiting
planet's RV semi-amplitude at a desired level of precision given the properties
of the planet, host star, and RV spectrograph. The properties of the latter two
are required, in part, to compute the photon noise limited RV precision using
model spectra from which the RV information content can be quantified. However,
late K to M dwarf model spectra are known to be incomplete largely due to the
abundance of molecular transitions that are not well understood
\citep{passegger16,behmard19}. Updating the M dwarf model spectra within the
\texttt{RVFC}, from which $\sigma_{\gamma}$ values are calculated, would improve
the accuracy of the \texttt{RVFC} calculations for cool stars. \\

Another limiting factor to the accuracy of computed values of $\sigma_{\gamma}$
is the treatment of telluric contamination. Currently in the \texttt{RVFC},
regions of the spectral domain for which the vertically integrated atmospheric
transmission is below some user-defined threshold are omitted from the
calculation. As noted in Sect.~\ref{sect:tellurics}, these regions are often
wide in the infrared such that calculations of $\sigma_{\gamma}$ in near-IR
bands are likely underestimated relative to the values that could be obtained
if a more sophisticated treatment of telluric contaimination was applied
\citep[e.g.][]{artigau14,bedell19}. Implementation of an accurate model of the
telluric spectrum and its subsequent subtraction from the observed spectrum
would signficantly increase the usable amount of spectral information for stars
observed with near-IR spectrographs and would thus decrease values of
$\sigma_{\gamma}$ in the near-IR closer to typical values in the optical (see
Fig.~\ref{fig:RVFCfig:sigRV}). \\

Another point of potential improvement to the accuracy of the \texttt{RVFC} is
to allow for custom window functions to be used. In the special case of planets
plus white noise in RV time series, exact window function does not affect the
$K$ measurement precision \sigK{} as long as the time series approximately
samples the planet's orbital phase uniformly. However when correlated noise
persists, the form of the window function matters because of the factors
$t_i-t_j$ in the quasi-periodic covariance function. Currently the \texttt{RVFC}
adopts a uniformly spaced window function for correlated noise calculations but
the option for users to define their own window functions that take into account
target visibility and/or observatory resistrictions would be desirable.


\section{Improving our Physical Understanding of M dwarf Stellar Activity}
The Sun is unique in that its surface can be resolved with a variety of
observational diagnostics of its brightness, velocity, magnetic field strength,
and orientation distributions. When those observables are obtained
simultaneously with other time series that are obtainable for distant stars
(e.g. disk integrated photometry and RVs), then resolved structures on the solar
surface can be related to the other observables in inform models of solar
activity \citep{dumusque15,haywood16}. Numerous programs which I have recently
become involved with\footnote{The RVxTESS collaboration which has proposed
  to use a number of optical and near-IR RV spectrographs and polarimeters to
contemporaneous observe TESS sectors } aim to observe a sample of M dwarfs with a similar
strategy. That is, to obtain contemporaneous optical and near-IR RVs along with
polarimetric diagnostics of bright M dwarfs spanning a range of stellar masses
and rotation periods. This large collaboration known as the RVxTESS program
has proposed to obtain those observations of M dwarfs within a suite of TESS
sectors as they are being observed with TESS. The goal of this and other
similar programs is to provide as many contemporaneous diagnostics of M dwarf
stellar activity in order to infer the physical nature of activity signals that
are seen in the RVs and that hinder our ability to detect low mass planets. \\

%In the sense that all models are incomplete a better understanding of the
%physical sources of stellar activity on M dwarfs will help inform any absent
%components in our RV models. 
Deepening our understanding of the physical sources of M dwarf stellar may also
be useful for modifying our GP models of stellar activity in that additional
covariance terms may be required or these correlated noise models may need to
be supplemented with other physical models of RV activity
\citep{aigrain12,haywood14}. Another potential
outcome of these studies is the establishment of what observational diagnostics
are optimal in terms of informing RV activity models as I have discerned that
in practice, stellar photometry provides the strongest constraints on RV
activity despite the photometric time series used throughout
chapters~\ref{chap:k2181} and~\ref{chap:k2182} were not contemporaneous with
the RV measurements.


\section{Detailing the Accuracy of RV Planet Models in the Presence of a GP}
Since some of the original implementations of GP models of correlated noise in
RV time series the results have been widely accepted due to the ability of the
formalism to return precise planetary parameters in models that are favoured
over models lacking such a treatment of stellar activity
\citep{haywood14,rajpaul15}. What remains to be quantified is the accuracy of
derived planetary parameters from RV models containing a GP activity model. For
example, the GP treatment of stellar activity is known to at times absorb some
of the planetary signal \citep[e.g.][]{ribas18}. Furthermore, GP activity models
that are untrained are at times disfavoured over models with trained GP models
as they lack the
constraining power provided by ancillary activity-sensitive time series like
photometry \citep{cloutier17b}. \\

One avenue worth considering to quantify the accuracy of planet parameters
derived from
RV models containing a GP activity model is to compute planet parameters
from a suite of synthetic RV time series for which the injected planet
parameters are known. Calculating the bias function as the difference between
the maximum a-posteriori $K$ (for example) and the injected $K$ for models with
and without a GP would inform the ways in which the GO biases our inference of
observable planetary parameters and consequently their physical properties.
Inherent biases in the planetary mass-radius relation resulting from only
reporting high S/N planet mass detections has already been noted \citep{burt18}.
Whether or not the incompleteness of GP activity models is resulting in biased
planetary mass measurements, we want known about it.


\section{Characterizing the Performance of \texttt{ORION}}
Although the TESS mission was not designed as a statistical mission to study
the bulk properties of the exoplanet population and subsets therein,
characterization of the sensitivity and false positive rate of transit detection
algorithms will enable the planet occurrence rates to be derived. Particularly,
for close-in planets where TESS is most sensitive due its 27 day fields.
However, these planetary systems are of particular interest given structure
that was recently detected in the occurrence rate of close-in planets around
Sun-like stars postulated to be the result of photoevaporation on the planetary
atmospheres \cite{owen13,fulton17,vaneylen18}. \\

A thorough set of injection-recovery simulations using the \texttt{ORION}
transit detection code (see chapter~\ref{chap:orion}) may be used to
characterize the algorithm's detection sensitivity and systematic false positive
rate as a function of planetary orbital period and $r_p/R_s$. Application to the
TESS data will then allow for the occurrence rate of close-in planets around M
dwarfs to be calculated as TESS features an enhanced sensitivity to early-to-mid
M dwarfs than did Kepler. Focusing on the detailed occurrence rates of these
planetary systems will enable the resolution of the radius valley for M dwarf
planetary systems and the comparision of its location and slope with orbital
period to the results obtained for Sun-like stars \citep{fulton18}.
