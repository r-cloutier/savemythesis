\clearpage
\begin{landscape}
\begin{table*}
  \small
  \renewcommand{\arraystretch}{0.7}
  \caption{Descriptions of the free parameters controlling the performance of \pipeline{}}
  \label{oriontable:freeparams}
  \begin{tabular}{clcl}
    \hline \\ [-1ex]
    Parameter & Definition & Default Value & Summary of Behavior \\
    \hline
    \multicolumn{4}{c}{\emph{Linear search parameters}} \\
    \hline
    $\Delta t$ & Linear search temporal bin width & 30 minutes &
    Decreasing $\Delta t$ will improve sensitivity to ultra-short period \\
    &&& planets while increasing the \pipeline{} runtime number of signals \\
    &&& for confusion in the periodic search stage. \\ 
    $D$ grid & Grid of transit durations considered during the & $\{1.2,2.4,$ &
    Decreasing the minimum duration improves sensitivity to \\
    & linear search stage & $4.8\}$ hours &
    ultra-short period planets but makes \pipeline{} more \\
    &&& susceptible to stochastic features with short timescales. \\ 
    S/N$_{\text{thresh}}$ & Minimum linear search S/N of a transit- & 5 &
    Decreasing S/N$_{\text{thresh}}$ will result in more light curve features \\
    & like event &&
    being flagged as false positives thus creating more signals for \\
    &&& confusion within the periodic search stage.  \\
    \multicolumn{4}{c}{\emph{Periodic search parameters}} \\
    \hline \\
    $f_{\text{P}}$ & Maximum relative difference between two  & 0.01 &
    Increasing $f_{\text{P}}$ makes fewer period pairs consistent with  \\
    & periods to be flagged as a multiple &&
    being multiples thus increasing the sensitivity to resonant \\
    &&& planet pairs while simultaneously increasing the number \\
    &&& of single planets misidentifed as a resonant pair.  \\
    \multicolumn{4}{c}{\emph{Automated light curve vetting parameters}} \\
    \hline \\
    $c_1$ & Minimum transit S/N & 8.4 & Increasing $c_1$ prevents the detection of some small planets \\
    &&& but will also significantly reduce the number false positives \\
    &&& due to residual systematics. \\
    $c_2$ & Minimum number of MADs from the & 2.4 &
    Similar behavior to $c_1$. \\
    & out-of-transit flux that the difference in && \\
    & median in and out-of-transit fluxes  && \\
    & must exceed && \\
    $c_3$ & Minimum fraction of in-transit points & 0.7 &
    Increasing $c_3$ may result in more accurate determinations of \\
    & below $Z+\sigma_Z$ &&
    correct periods but will also cause some transits to be \\
    &&& discarded if residual noise is also present during the transit.  \\
    $c_4$ & Minimum fraction of in-transit points & 0.1 &
    Increasing $c_4$ increases sensitivity to asymmetric transit \\
    & prior to $T_0$ && 
    shapes such as those from disintegrating planets. \\
    $c_5$ & Minimum number of MADs for a flare & 8 &
    Increasing $c_5$ makes flare detection smore robust but at the \\
    & above the flux continuum &&
    risk of missing some lower amplitude flares. \\
    $c_6$ & Minimum number of successive points & 2 &
    Increasing $c_6$ decreases sensitivity to flares of short relative \\
    & within a flare duration & & to the light curve cadence. \\
    $c_7$ & Assumed M dwarf flare duration & 30 minutes &
    Increasing $c_7$ decreases the sensitivity to long duration flares \\
    &&& while increasing the assumed fraction light curve fraction that \\
    &&& is contaminated by flares. \\
    $c_8$ & Number of transit durations from $T_0$ & 4 &
    Decreasing $c_8$ decreases the probability of an transit being \\
    & which cannot be affected by flares &&
    contaminated by flares. \\
    $c_9$ & Maximum time from the light curve & 4.8 hours &
    Increasing $c_9$ descreases the probability of a transit being \\
    & edges to not be rejected due to possible &&
    affects by light curve systematics at its edges but also narrows \\
    &  contamination at the edges. &&
    the baseline over which transits can be found. \\
    $c_{10}$ & Minimum autocorrelation of flux residuals & 0.6 &
    Increasing $c_{10}$ improves the robustness of transit detections \\
    &&& but decreases the detection sensitivity in light curves with \\
    &&& imperfect systematics corrections. \\
    \multicolumn{4}{c}{\emph{Eclipsing binary vetting parameters}} \\
    \hline \\
    $c_{\text{EB},1}$ & Maximum $r_p/R_s$ of a  transit-like event & 0.5 & - \\
    $c_{\text{EB},2}$ & Maximum planet radius & 30 R$_{\oplus}$ & - \\
    $c_{\text{EB},3}$ & Maximum transit duration, & $D(30\text{ R}_{\oplus},$ & - \\
    & $D(r_p,P,a/R_s,Z,i)$ & $P,a/R_s,Z,i)$ & \\
    $c_{\text{EB},4}^{\text{a}}$ & Minimum occultation S/N of an EB (Eq.~\ref{eq:occ}) & 5 &
    Decreasing $c_{\text{EB},4}$ makes a larger fraction of occultations \\
    &&& consistent with being due to an EB rather than a transiting \\
    &&& planet. \\
    $c_{\text{EB},5}^{\text{a}}$ & Minimum fraction of iterative occultation & 0.5 &
    Increasing $c_{\text{EB},5}$ makes transit-like events more robust as \\
    & searches consistent with an EB & &
    the probability of being flagged as an EB is reduced. \\ 
    $c_{\text{EB},6}^{\text{a}}$ & Minimum ingress plus egress time fraction & 0.9 &
    Increasing $c_{\text{EB},6}$ makes fewer transit-like events consistent \\
    & of $D$ for a `V'-shaped transit && with having a `V'-shaped transit. \\
  \end{tabular}
  \begin{list}{}{}
  \item {\bf{Notes.}}
  \item $^{\text{a}}$ These eclising binary (EB) parameters are intended to identify favorable EBs rather than transiting planets. Planetary signals of interest are rejected if any of the EB criteria are satisfied.
    \end{list}
\end{table*}
\clearpage
\end{landscape}
