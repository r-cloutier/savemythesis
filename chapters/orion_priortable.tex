\begin{deluxetable*}{lc}
\tablecaption{Model parameter priors\label{table:priors}}
\tablehead{Parameter & Prior}
\startdata
\multicolumn{2}{c}{\emph{GP hyperparameters}\tablenotemark{a}} \\
Covariance amplitude, $\ln{a_{\text{GP}}}$ & $\mathcal{U}(-20,0)$ \\
Exponential timescale, $\ln{\lambda/}$days & $\mathcal{U}(-3,10)$ \\
Coherence, $\ln{\Gamma}$ & $\mathcal{U}(-5,5)$ \\
Periodic timescale, $\ln{P_{\text{GP}}/}$days & $\mathcal{U}(-3,10)$  \\
\multicolumn{2}{c}{\emph{Transit model parameters}} \\
Orbital period, $P$ [days] & $\mathcal{U}(0.9,1.11)\cdot P_{\text{opt}}$\tablenotemark{b} \\
Time of mid-transit, $T_0$ & $\mathcal{U}(-1.11,1.11)\cdot P_{\text{opt}} + T_{0,\text{opt}}$ \\
$[$BJD-2,457,000$]$ & \\
Scaled semimajor axis, $a/R_s$ & $\mathcal{U}(0.58,1.70)\cdot (a/R_s)_{\text{opt}}$ \\
Planet-star radius ratio, $r_p/R_s$ & $\mathcal{U}(0,1)$ \\
Orbital inclination, $i$ & $\mathcal{U}(-1,1)\cdot i((a/R_s)_{\text{opt}},b=1)$\tablenotemark{c} \\
\multicolumn{2}{c}{\emph{Single transit model parameters}} \\
Orbital period, $P$ [days] & $\mathcal{J}(1,100)\cdot P_{\text{inner}}$\tablenotemark{d} \\
Time of mid-transit, $T_0$ & $\mathcal{U}(-3,3)\cdot D$ \\
$[$BJD-2,457,000$]$ & \\
Scaled semimajor axis, $a/R_s$ & $\mathcal{J}(1,100)\cdot (a/R_s)_{\text{inner}}$ \\
Planet-star radius ratio, $r_p/R_s$ & $\mathcal{U}(0,1)$ \\
Orbital inclination, $i$ & $\mathcal{U}(-1,1)\cdot i((a/R_s)_i,b=1)$ \\
\enddata
\tablenotetext{a}{GP hyperparameter priors used during de-trending (i.e. with zero mean model) and during the simultaneous systematics plus transit modeling.}
\tablenotetext{b}{The designation `opt' is indicative of the 
optimized parameter values from the maximum likelihood model used for parameter initialization.}
\tablenotetext{c}{The function $i(a/R_s,b)=a \cos{i}/R_s$ returns the orbital inclination
  given $a/R_s$ and the impact parameter $b$ which is constrained to $|b|<1$ in our
  transit models.}
\tablenotetext{d}{The designation `inner' is indicative of the inner-most orbital period 
permissible for a single transit event over the \tess{} baseline.}
%\tablecomments{Note that {\tt \string \colnumbers} does not work with the
%vertical line alignment token. If you want vertical lines in the headers you
%can not use this command at this time.}
\end{deluxetable*}
